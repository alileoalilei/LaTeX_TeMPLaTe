%%%%%%%%%%%%%%%%%%%%%%%%%%%%%%%%%%%%%%%%%
% a0poster Portrait Poster
% LaTeX Template
% Version 1.0 (22/06/13)
%
% The a0poster class was created by:
% Gerlinde Kettl and Matthias Weiser (tex@kettl.de)
% 
% This template has been downloaded from:
% http://www.LaTeXTemplates.com
%
% License:
% CC BY-NC-SA 3.0 (http://creativecommons.org/licenses/by-nc-sa/3.0/)
%
%%%%%%%%%%%%%%%%%%%%%%%%%%%%%%%%%%%%%%%%%

%----------------------------------------------------------------------------------------
%	PACKAGES AND OTHER DOCUMENT CONFIGURATIONS
%----------------------------------------------------------------------------------------

\documentclass[a0,portrait]{a0poster}

\usepackage{multicol} % This is so we can have multiple columns of text side-by-side
\columnsep=50pt % This is the amount of white space between the columns in the poster
\columnseprule=3pt % This is the thickness of the black line between the columns in the poster

\usepackage[svgnames]{xcolor} % Specify colors by their 'svgnames', for a full list of all colors available see here: http://www.latextemplates.com/svgnames-colors

\usepackage{times} % Use the times font
%\usepackage{palatino} % Uncomment to use the Palatino font
%\usepackage{helvet}

\usepackage{graphicx} % Required for including images
\graphicspath{{figures/}} % Location of the graphics files
\usepackage{booktabs} % Top and bottom rules for table
\usepackage[font=small,labelfont=bf]{caption} % Required for specifying captions to tables and figures
\usepackage{amsfonts, amsmath, amsthm, amssymb} % For math fonts, symbols and environments
\usepackage{wrapfig} % Allows wrapping text around tables and figures
\usepackage{tcolorbox}
\usepackage{caption}
\definecolor{mblue}{rgb}{0,0.447,0.741}
\definecolor{morange}{rgb}{0.85,0.325,0.098}

\def \mytitle{Enter the poster title here}
\def \authors{Enter the name of the authors here}
\def \contact{Enter the contact information here}
\def \institu{Enter the name of the institute here}

\begin{document}

%----------------------------------------------------------------------------------------
%	POSTER HEADER 
%----------------------------------------------------------------------------------------

% The header is divided into two boxes:
% The first is 75% wide and houses the title, subtitle, names, university/organization and contact information
% The second is 25% wide and houses a logo for your university/organization or a photo of you
% The widths of these boxes can be easily edited to accommodate your content as you see fit
\begin{minipage}[b]{0.8\linewidth}
\begin{center}
\Huge \color{FireBrick} \textbf{\mytitle} \color{Black}\\ % Title
\end{center}

\begin{center}
\LARGE \textbf{\authors}\\[0.25cm] % Author(s)
\Large \institu \\[0.1cm] % University/organization
\large \texttt{\contact}\\
\end{center}
\end{minipage}
%
\begin{minipage}[b]{0.2\linewidth}
	% a logo here
\end{minipage}

\noindent\makebox[\linewidth]{\rule{\linewidth}{5.6pt}}
%----------------------------------------------------------------------------------------

\begin{multicols}{3} \setlength{\columnseprule}{0pt}
% This is how many columns your poster will be broken into, a portrait poster is generally split into 2 columns

%----------------------------------------------------------------------------------------
%	ABSTRACT
%----------------------------------------------------------------------------------------

%\begin{abstract}

%\end{abstract}

%----------------------------------------------------------------------------------------
%	INTRODUCTION
%----------------------------------------------------------------------------------------

\color{DarkSlateGray} % SaddleBrown color for the introduction

\section*{Motivation}
Write the motivation here
\linebreak
%----------------------------------------------------------------------------------------
%	OBJECTIVES
%----------------------------------------------------------------------------------------
\begin{tcolorbox}
\Large  \textbf{OBJECTIVES}
\begin{itemize}
\item  itemize the objectives here
\end{itemize}
\end{tcolorbox}
%----------------------------------------------------------------------------------------
%	MATERIALS AND METHODS
%----------------------------------------------------------------------------------------
%------------------------------------------------
\color{DarkSlateGray} 
\section*{Another Section}

\begin{itemize}\small
 \item Model setup etc...
\end{itemize}
 
\section*{Another section}
\begin{center}%\vspace{1cm}
	% a figure here
\end{center}%\vspace{1cm}

\begin{itemize}\small
\item 
\end{itemize}

\end{multicols}


%----------------------------------------------------------------------------------------
%	RESULTS 
%----------------------------------------------------------------------------------------
\noindent\makebox[\linewidth]{\rule{\linewidth}{5.6pt}}

\begin{multicols}{3}\setlength{\columnseprule}{0pt}
\section*{Another section here}
\begin{center}
	% a figure here
\end{center}%\vspace{1cm}

\begin{center}
	% you can include figures here
\end{center}


\begin{tcolorbox}[colback=mblue!5!white]
\Large \textbf{A title here}
\begin{itemize}
\item
\end{itemize}
\end{tcolorbox}

\end{multicols}

\begin{multicols}{3}\setlength{\columnseprule}{0pt}
 \begin{tcolorbox}[colback=morange!5!white]
 \Large  \textbf{A title here}
 \begin{itemize}
 \item 
 \end{itemize}
 \end{tcolorbox}

 \begin{center}
	 % a figure here
 \end{center}
 
 \begin{center}
	 % a figure here
 \end{center}


\end{multicols}
\begin{tcolorbox}[colback=morange!2!white,colframe=morange!100!white,title=\Huge Discussions \& Future work,boxsep=5mm]
\begin{multicols}{3}\setlength{\columnseprule}{0pt}
  \begin{itemize}
  \item 
  \end{itemize}
   \begin{tcolorbox}[colback=mblue!15!white]
   \textbf{May be a future work}
  \begin{itemize}
  \item 
  \end{itemize}
  \end{tcolorbox}
  \textbf{Acknowledgements}
  %\nocite{*} % Print all references regardless of whether they were cited in the poster or not
  %\bibliographystyle{plain} % Plain referencing style
  %\bibliography{sample} % Use the example bibliography file sample.bib
\end{multicols}
\end{tcolorbox}
\end{document}
